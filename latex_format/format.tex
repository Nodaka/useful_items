\documentclass{jarticle}
\usepackage[utf8]{inputenc}
\usepackage{amsmath}
\usepackage{mathtools}
%\usepackage{indentfirst}%文の最初をインデントする.
%\setlength{\parindent}{12pt}%最初の字下げ幅を設定できる.
\usepackage[top=25truemm,bottom=20truemm,left=50truemm,right=50truemm]{geometry}%余白の調整

\title{タイトル}
\author{名前}
\date{日付}
\begin{document}
\maketitle
\section{章}

indentfirstを設定すると最初の1行目も字下げした状態で文章が始まり, 日本の形式にあった表現になる.

2行目は以降は自動的に字下げされるが, これは段落改行という行と行の間に空白行を挿入することによる改行の場合は適応される.\\
なので, バックスラッシュを2連でつなげた強制開業の場合は字下げが起こらないので注意が必要.

\begin{enumerate}
  \item 箇条書き1
  \begin{enumerate}
    \item 箇条書き1-1
    \begin{enumerate}
      \item 箇条書き1-1-1
      \item 箇条書き1-1-2
    \end{enumerate}
    \item 箇条書き1-2
  \end{enumerate}
  \item 箇条書き2
\end{enumerate}
\subsection{節}
\subsubsection{項}

\begin{equation} \label{eq:a}
  f(x) = \sin x
\end{equation}

\begin{align}
  f(x) &= \sin x  \\
  g(x) &= \cos (-x) \label{eq:cos} \\
  h(x) &= \tan(1/x)
\end{align}

式\eqref{eq:cos}を参照する.
\subsubsection{数式の例}
\end{document}