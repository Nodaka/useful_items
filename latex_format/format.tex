\documentclass{article}
\usepackage[utf8]{inputenc}
\usepackage{amsmath}
\usepackage{mathtools}
\title{サンプルレポート}
\author{あなたの名前}
\date{\today}
\begin{document}
\maketitle
\section{はじめに}
\begin{enumerate}
  \item 半導体業界内で自由に転職できる実力あるエンジニア
  \begin{enumerate}
    \item 外資系企業に就職できる十分な英語力
    \begin{enumerate}
      \item TOEIC 860
      \item 英会話
    \end{enumerate}
    \item 半導体業界における十分な知見
    \begin{enumerate}
      \item 業界の本を読む
      \item 技術雑誌を読む
    \end{enumerate}
    \item 汎用的な技術知見
    \begin{enumerate}
      \item プログラミング
      \item 統計学の知見
      \item 物理知識
      \item 数学知識
      \item 研究知見
    \end{enumerate}
  \end{enumerate}
\end{enumerate}
\section{内容}
は文章やレポートの作成に便利です。数式も簡単に挿入できます。
\subsection{数式の例}
以下は二次方程式の解の公式です。どうでしょうか.
\begin{equation} \label{eq:a}
  f(x) = \sin x
\end{equation}

\begin{align}
  f(x) &= \sin x  \\
  g(x) &= \cos (-x) \label{eq:cos} \\
  h(x) &= \tan(1/x)
\end{align}

式\eqref{eq:cos}は
\subsubsection{数式の例}
\end{document}